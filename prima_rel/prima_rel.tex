\documentclass[a4paper, 11pt]{article}
\usepackage{comment} 
\usepackage{fullpage}
\usepackage[colorlinks,allcolors={blue}]{hyperref}
\usepackage{graphicx}
\begin{document}
	\hrulefill
	\begin{flushright}
		\textbf{Universit\`a di Padova - Dipartimento Fisica e Astronomia}\\ 
		\textbf{Corso:} Sperimentazioni 1 - Canale M-Z.\\
		\textbf{Anno accademico:} 2021-22.\\
		\textbf{Docenti:} D. Mengoni (\url{daniele.mengoni@unipd.it}) M.~Doro (\url{michele.doro@unipd.it}) \\
	\end{flushright}
	
	\bigskip
	\noindent
	%\noindent\makebox[\linewidth]{\rule{\paperwidth}{0.4pt}}
	\large{\textbf{Gruppo G22} \\
		Niccolo' Cesare Sartori - 2067047 - \url{niccolocesare.sartori@studenti.unipd.it} \\
		Nome Cognome 2 - Matricola - \url{indirizzo@e-mail} \\
		Nome Cognome 3 - Matricola - \url{indirizzo@e-mail} \\
		\begin{verbatim}Data consegna relazione: ___/___/______\end{verbatim} \hrulefill 
		
		
		
		
		\section*{Misurazione della costante di accelerazione gravitazionale tramite carrello su piano inclinato}

		
		\section[Metodologia]{Descrizione dell'apparato strumentale, della messa in opera e della procedura di misura}
		\emph{[Circa 2 pagine]}\\
		\begin{enumerate}
			\item Descrivere in modo sintetico l'apparato sperimentale utilizzato in relazione alla misura che si vuole fare, anche con l'aiuto di uno schema grafico utilizzando i \emph{concetti di metrologia} discussi in lezione. 
			\item Discutere quelle che sono le \emph{misure di input}, le \emph{misure output}, e il \emph{modello matematico} che porta alla stima del misurando.
			\item Discutere i \emph{fattori che influenzano la misura} e la \emph{sensitivit\`a presunta o misurata dello strumento} verso questi fattori.
			\item Descrivere la \emph{procedura di misura}
		\end{enumerate}
		
		
		Alcuni consigli e raccomandazioni:
		\begin{itemize}
			\item Se possibile, usare uno schema grafico com in Fig.~\ref{fig:newton} dell'apparato, piuttosto che una foto . Limitarsi ai parametri principali dello strumento in relazione alla misura (es. colore non e' importante). Il numero di immagini in questa sezione non puo' superare poche unit\`a. 
			\item nel caso di incertezze sistematiche discutere sempre se sono eliminabili o misurabili
		\end{itemize}
		
		Descrivere le procedure di messa in opera, calibrazione e regolazione del sistema di misura.
		
		Considerate che stiate descrivendo l'esperienza non ad una persona che gi\`a la conosce ma ad un possibile lettore esterno o un collega che vuole farsi una chiara idea della esperienza, ed eventualmente ripeterla. 
		
		
		
		
		\section[Analisi]{Presentazione dei dati, analisi e presentazione dei risultati}
		\textit{[Massimo circa 4 pagine]}\\
		In questa parte:
		\begin{itemize}
			\item Presentare i dati grezzi adeguatamente raggruppati in opportune tabelle numerate (vedi ad esempio Tabella~\ref{tab:mia_tabella}. La discussione che si fa sui dati qui \`e specifica, a questa se ne pu\`o aggiungere una seconda pi\`u generica nel capitolo discussione/conclusioni. 
			\begin{table}[h!t]
				\centering
				\begin{tabular}{c|ccc}
					\hline
					Misure Periodo  & T1-100  & T2-50 & T4-50\\
					\hline
					1 & 2.1 & 4.1 & 8.1\\
					2 & 2.2 & 4.2 & 8.2\\
					\hline
				\end{tabular}
				\caption{Tabella dati misura XX. }
				\label{tab:mia_tabella}
			\end{table}
			\item Verificare se necessario la presenza di \emph{outliers}. Nel caso di esclusioni, motivarne le ragioni.
			\item Specificare la procedura utilizzata per l'analisi dei dati includendo le formule usate e il significato degli elementi della formula. In generale, l'incertezza ottenuta per propagazione va riportata in relazione.
			
			%Ad esempio per l'incertezza sul modulo di Young si \`e usata al seguente espressione: 
			%\bigskip
			%\begin{equation}\label{eq:stderr}
			%   \sigma_{\overline{x}}=\frac{\sqrt{\frac{1}{N-1}\sum_{i=1}^N\;(x_i-\overline{x})^2}}{\sqrt{N}}
			%\end{equation}
			%\emph{dove $\overline{x}$ \`e la media aritmetica, $N$ numero delle misure, $x_i$ la singola misura.}  
			%\bigskip
			
			\item Solo nel caso in cui ci si riferisca a formule generali come ad esempio medie, incertezze della media etc., \`e preferibile riportare le suddette in appendice. 
			
			%Ad esempio: 
			%\bigskip
			%\emph{\ldots come incertezza si \`e usato errore standard dato da Eq.~\ref{eq:stderr}:}
			%\begin{equation}\label{eq:stderr}
			%    \sigma_{\overline{x}}=\frac{\sqrt{\frac{1}{N-1}\sum_{i=1}^N\;(x_i-\overline{x})^2}}{\sqrt{N}}
			%\end{equation}
			%\emph{dove $\overline{x}$ \`e la media %aritmetica, $N$ numero delle misure, $x_i$ la singola misura.}  
			%\bigskip
			
			Ad esempio: 
			\bigskip
			\begin{equation}\label{eq:stderr}
				\sigma_{\overline{x}}=\frac{\sqrt{\frac{1}{N-1}\sum_{i=1}^N\;(x_i-\overline{x})^2}}{\sqrt{N}}
			\end{equation}
			%\emph{dove $\overline{x}$ \`e la media aritmetica, $N$ numero delle misure, $x_i$ la singola misura.}  
			\bigskip
			
			\item Presentare i risultati sperimentali  in modo organico, cio\`e tabelle, grafici ed altro vanno spiegati contestualmente al loro inserimento in relazione.
			
			\item Ogni grandezza fisica deve esser accompagnata dal relativo errore e deve esser esposta l'unit\`a di misura, scritta in modo coerente e con il numero di cifre significative corrette.
			\item Per il corso di SF1 si chiede di includere la tabella dei dati grezzi per permettere al docente di verificare la correttezza dell'analisi dati. Nel caso in cui i dati acquisiti siano in numero elevato, si pu\`o omettere la tabella dei dati grezzi, che andranno presentati in modo esemplificativo in appendice, ma in tal caso \`e importante almeno presentare in sostituzione uno o pi\`u grafici delle grandezze fisiche in gioco per mostrare i relativi andamenti come ad esempio il grafico della distribuzione temporale dei dati come in Fig.~\ref{fig:evol}.
			
			
		\end{itemize}
		
		\section[Discussione]{Discussione dei risultati sperimentali e conclusioni.}
		\emph{[Massimo circa 2 pagine]}\\
		Questa sezione si divide in 3 parti concettuali:
		\begin{enumerate}
			\item Riassunto risultati principali
			\item Discussione risultati
			\item Sguardo in avanti: come migliorare risultato
		\end{enumerate}
		
		Nella prima parte, riportare in forma breve e concisa i risultati numerici della esperienza in forma riassunta. Un lettore dovrebbe essere in grado di giudicare la vostra relazione solo dalla discussione e dalle conclusioni. Successivamente discutere criticamente questi risultati, anche in relazione agli obiettivi posti nella introduzione. Concludere con una discussione delle operazioni che sarebbero suggeribili per migliorare la precisione e/o l'accuratezza del risultato, sempre che sia  possibile e applicabile.
		
		In aggiunta, si valuti a posteriori delle incertezze sistematiche dello strumento e della misura, della loro possibile eliminazione. Le conclusioni devono esser coerenti con gli obiettivi preposti.
		
		\paragraph{Appendici} 
		\emph{[Massimo circa 5 pagine]}\\
		In appendice si riporta tutto quello che in prima lettura non e' necessario a dimostrare le conclusioni trovate durante l'esperienza, ma che pu\`o tornare utile in caso di un controllo successivo o un esame approfondito. Ad esempio in appendice si possono riportare i codici usati, tabelle troppo lunghe, grafici di controllo. Anche in questo caso usare moderazione.
		
	\end{document}